\documentclass[10pt, twoside]{book}

%%%%%%%%%
% Maths %
%%%%%%%%%

\usepackage{math-fonts}
\usepackage{math-graphics}
\usepackage{math-symbols}
\usepackage{math-theorems}

%%%%%%%%%
% Title %
%%%%%%%%%

\title{Lecture Notes to Functional Analysis \\ \large{Winter 2020, Technion}}
\author{Lectures by Liran Rotem \\ \small{Typed by Elad Tzorani}}
\date{\today}

\usepackage{hyperref}

\begin{document}

\maketitle
\tableofcontents

\chapter{Preliminaries}

\section{Banach Spaces \& Examples}

\begin{definition}[Norm]
Let $X$ be a vector space over $\mbb{F} \in \set{\mbb{R}, \mbb{C}}$ (and assume from now until mentioned otherwise these are the only fields).
A \emph{norm} on $X$ is a function
\[\norm{\cdot} \colon X \to \left[0, \infty \right)\]
such that the following hold.
\begin{enumerate}
\item $\norm{x} = 0$ iff $x = 0$.
\item $\norm{\lambda x} = \abs{\lambda} \norm{x}$ for all $\lambda \in \mbb{F}$.
\item $\norm{x+y} \leq \norm{x} + \norm{y}$.
\end{enumerate}
\end{definition}

\begin{remark}
A norm $\norm{\cdot}$ defines a metric by $d\prs{x,y} = \norm{x-y}$. This defines a topology generated by the open balls in the metric.
\end{remark}

\begin{definition}[Banach Space]
A normed space $\prs{X, \norm{\cdot}}$ is a \emph{Banach space} if it's complete (i.e. if every Cauchy sequence in it converges).
\end{definition}

\begin{definition}[Inner Product]
Let $X$ be a vector space over $\mbb{F}$.
An \emph{inner product on $X$} is a function
\[\trs{\cdot, \cdot} \colon X \times X \to \mbb{F}\]
such that the following hold.
\begin{enumerate}
\item $\trs{\lambda x_1 + x_2, y} = \lambda \trs{x_1, y} + \trs{x_2, y}$.
\item $\trs{x,y} = \overline{\trs{y,x}}$.
\item $\trs{x,x} \geq 0$ and $\trs{x,x} = 0$ iff $x = 0$.
\end{enumerate}
\end{definition}

\begin{remark}
An inner product $\trs{\cdot, \cdot}$ defines a norm $\norm{x} = \sqrt{\trs{x,x}}$. This can be proven by Cauchy-Schwarz $\abs{\trs{x,y}} \leq \sqrt{\norm{x} \norm{y}}$.
\end{remark}

\begin{definition}[Hilbert Space]
A \emph{Hilbert space} is a complete inner product space.
\end{definition}

\begin{example}[$\ell_p$ Spaces]
Fix $p \in \left[1, \infty \right)$ and $\prs{a_n}_{n=1}^{\infty}$ a sequence in $\mbb{F}$. Define
\begin{align*}
\norm{\prs{a_n}_{n=1}^{\infty}}_p &\ceq \prs{\sum_{n=1}^{\infty} \abs{a_n}^p}^{\frac{1}{p}} \\
\norm{\prs{a_n}_{n=1}^{\infty}}_\infty &\ceq \sup_{n \in \mbb{N}_+} \abs{a_n}
\end{align*}
and
\begin{align*}
\ell_p &\ceq \set{\prs{a_n}_{n=1}^{\infty}}{\norm{\prs{a_n}_{n = 1}^{\infty}}_p < \infty} \\
\ell_\infty &\ceq \set{\prs{a_n}_{n=1}^{\infty}}{\norm{\prs{a_n}_{n=1}^{\infty}}_\infty < \infty}
\end{align*}
\end{example}

\begin{example}
Define
\[c \ceq \set{\prs{a_n}_{n \in \mbb{N}_+} \in \ell_\infty}{\exists \lim_{n \to \infty} a_n \in \mbb{C}} \text{.}\]
Consider this with the $\infty$-norm, as a closed subspace of $\ell_\infty$.

Similarly, define
\[c_0 \ceq \set{\prs{a_n}_{n \in \mbb{N}_+} \in \ell_\infty}{\lim_{n \to \infty} a_n = 0} \text{.}\]
\end{example}

\begin{example}[Continuous Functions over a Compact Space]
Let $X$ be a compact topological space.
Define
\[\mscr{C}\prs{X} \ceq \set{f \colon X \to \mbb{F}}{f \text{ is continuous}} \text{.}\]
This is a Banach space with $\norm{f} = \max_{x \in X} \abs{f\prs{x}}$.
\end{example}

\begin{definition}[Support of a Function]
Let $f \colon X \to V$ be a function of sets from a topological space into a vector space. Define \emph{the support of $f$} to be
\[\sup \prs{f} \ceq \overline{\set{x}{f\prs{x} \neq 0}} \text{.}\]
\end{definition}

\begin{example}[Continuous Functions over Locally Compact Hausdorff Spaces]
Let $X$ be a locally compact Hausdorff topological space (e.g. $X = \mbb{R}^n$).
Define
\[\mscr{C}_c\prs{X} \ceq \set{f \colon X \to \mbb{F}}{f \text{ is compactly supported and continuous}} \text{.}\]
This is a normed space which is \emph{not} complete.

Define $f\prs{x} = e^{-x^2}$. This can be approximated by continuous functions $f_n$ which agree with it on $\brs{-n,n}$ and go linearly to zero outside $\brs{-n,n}$ until the boundary of $\brs{-n-1, n+1}$. $\prs{f_n}_{n = 1}^{\infty} \subseteq \mscr{C}_c\prs{X}$ is Cauchy but not convergent because $f \notin \mscr{C}_c\prs{X}$.
\end{example}

\begin{definition}[Completion of a Nםרצקג Space]
Let $Y$ be a normed space. There is a Banach space $\hat{Y}$ such that $Y \leq \hat{Y}$ and $\bar{Y} = \hat{Y}$. This is called \emph{the completion of $Y$}.
\end{definition}

\begin{example}
\[\widehat{\mscr{C}_c\prs{X}} = \mscr{C}_0\prs{X} \ceq \set{f \colon X \to \mbb{F}}{\substack{\text{$f$ is continuous} \\ \lim_{\abs{x} \to \infty} f\prs{x} = 0}} \text{.}\]
\end{example}

\begin{example}
Let $\prs{X, \Sigma, \mu}$ be a measure space and let $p \in \left[1, \infty \right)$.
We define
\[L_p\prs{X, \Sigma, \mu} \ceq \set{f \colon X \to \mbb{F}}{\substack{\text{$f$ is Borel measurable} \\ \norm{f}_p < \infty}}\]
where
\[\norm{f}_p \ceq \prs{\int_X \abs{f}^p \diff \mu}^{\frac{1}{p}}\]
and where we take the Borel measure on $\mbb{F}$.
\end{example}

\begin{remark}
If $X = \mbb{N}_+$ and $\mu$ is the counting measure $\mu\prs{a} = \abs{A}$ then $L_p\prs{X, \Sigma, \mu} = \ell_p$.
\end{remark}

\begin{remark}
$\norm{\cdot}_p$ which we defined isn't \emph{exactly} a norm, since there are measurable functions $f \colon X \to \mbb{F}$ which are $0$ almost everywhere, but not everywhere.
\\
We therefore look at $L_p$ as the space of equivalence classes of functions up to equivalence almost-everywhere of functions.
\end{remark}

On $\prs{X, \Sigma, \mu}$, $\norm{\cdot}_p$ is a norm. This follows from the following inequality.

\begin{theorem}[Hölder Inequality]
If $p,q \geq 1$ and $\frac{1}{p} + \frac{1}{q}$ then for all $f,g \colon X \to \mbb{F}$ it holds that
\[\abs{\int_X fg \diff \mu} \leq \prs{\int_X \abs{f}^p \diff \mu}^{\frac{1}{p}} \prs{\int_X \abs{g}^q \diff \mu}^{\frac{1}{q}} \text{.}\]
\end{theorem}

\begin{corollary}
\[\norm{f+g}_p \leq \norm{f}_p + \norm{g}_p \text{.}\]
\end{corollary}

\begin{corollary}
On $\prs{X, \Sigma, \mu}$, $\norm{\cdot}_p$ is a norm.
\end{corollary}

\begin{theorem}
$L_p\prs{X, \Sigma, \mu}$ is a Banach space.
\end{theorem}

\begin{proof}
It's enough to prove that if
\[\sum_{n=1}^{\infty} \norm{f_n}_p < \infty\]
then
$\sum_{n=1}^{\infty} f_n$ converges.

Assume first $f_n \geq 0$. Define $g \ceq \sum_{n=1}^{\infty} f_n \in \brs{0,\infty}$ and $g_N \ceq \sum_{n=1}^N f_n$.
We know by the triangle inequality that
\[\norm{g_N}_p \leq \sum_{n=1}^N \norm{f_n}_p \leq \sum_{n=1}^{\infty} \norm{f_n}_p \eqqcolon S \text{.}\]
Now
\[\norm{g}_p^p = \int_X \abs{g}^p \diff \mu = \int_X \lim_{N \to \infty} \abs{g_N}^p \diff \mu \underset{\text{Monotone Convergence}}{=} \lim_{N \to \infty} \int_X \abs{g_N}^p \diff \mu \leq S^p\]
hence $g \in L_p$.
In particular, $g < \infty$ almost-everywhere.

Finally,
\[\norm{g - g_N}_p = \norm{\sum_{n = N+1}^\infty f_n}_p \leq \sum_{n = N + 1}^\infty \norm{f_n}_p \xrightarrow{N \to \infty} 0 \text{.}\]

In general, we can write $f_n = f_n+ - f_n^{-}$ where $f_n^+, f_n^-$ are positive. We have $\abs{f_n^\pm} \leq \abs{f_n}$ so $\norm{f_n^\pm}_p \leq \norm{f_n}_p$.
So $\sum_{n=1}^{\infty} \norm{f_n^\pm}_p < \infty$ so $\sum_{n=1}^{\infty} f_n^\pm$ converge and so does $\sum_{n=1}^{\infty} \prs{f_n^+ - f_n^{-}}$.
\end{proof}

\begin{definition}
Let $f \colon X \to \mbb{F}$.
Define
\[\essup\prs{f} \ceq \inf \set{M > 0}{\mu\prs{\set{x}{\abs{f\prs{x}} \geq M}} = 0} \text{.}\]
\end{definition}

\begin{definition}[$L_\infty$]
Define
$\norm{f}_\infty = \essup\prs{f}$
and
\[L_\infty\prs{X, \Sigma, \mu} = \set{f \colon X \to \mbb{F}}{\norm{f}_\infty < \infty} \text{.}\]
\end{definition}

\begin{exercise}
Assume $\mu\prs{X} < \infty$.
\begin{enumerate}
\item If $1 \leq p < q \leq \infty$ then $L_q \subseteq L_p$.
\item If $f \in L_\infty \prs{X, \Sigma, \mu}$ then $\norm{f}_\infty = \lim_{p \to \infty} \norm{f}_p$.
\end{enumerate}
\end{exercise}

\begin{definition}[The Operator Norm]
Let $X,Y$ be normed spaces and let $T \colon X \to Y$ be linear.
Define the \emph{operator norm} as
\[\norm{T} = \sup_{x \in X \setminus \set{0}} \frac{\norm{T x}_Y}{\norm{x}_X} = \sup_{\substack{x \in X \\ 0 < \norm{x} \leq 1}} \norm{Tx} \text{.}\]
\end{definition}

\begin{definition}[Bounded Linear Map]
A linear map $T \colon X \to Y$ between normed spaces is \emph{bounded} if $\norm{T} < \infty$.
\end{definition}

\begin{fact}
Let $T \colon X \to Y$ be a linear map between normed spaces. The following are equivalent.
\begin{enumerate}
\item $T$ is bounded.
\item $T$ is continuous.
\item $T$ is continuous at $x = 0$.
\item $T$ is Lipschitz.
\end{enumerate}
\end{fact}

\begin{notation}
Let $X,Y$ be normed spaces.
We denote the class of bounded linear functions $X \to Y$ by $\mscr{L}\prs{X,Y}$.
\end{notation}

\begin{theorem}
If $Y$ is a Banach space, so is $\mscr{L}\prs{X,Y}$.
\end{theorem}

\begin{definition}[The Dual Space]
The \emph{dual space} of a normed space $X$ is $X^* \ceq \mscr{L}\prs{X,\mbb{F}}$.
\end{definition}

\begin{remark}
If $X$ is a Banach space, so is $X^*$.
\end{remark}

\begin{example}
Let $H$ be a Hilbert space. By the Riesz representation theorem we know that every $f \in H^*$ is of the form
\[f\prs{x} = f_y\prs{x} \ceq \trs{x,y}\]
for some $y \in H$.

The maps $y \mapsto f_y$ is a bijection, it holds that $\norm{f_y} = \norm{y}$, and
\[f_{\alpha y}\prs{x} = \trs{x, \alpha y} = \bar{\alpha} \trs{x,y} = \bar{\alpha} f_y\prs{x} \text{.}\]
\end{example}

\begin{example}
Let $1 \leq p < \infty$. It holds that $\prs{\ell_p}^* = \ell_q$ where $\frac{1}{p} + \frac{1}{q} = 1$.

For $c \ceq \prs{c_n}_{n \in \mbb{N}_+}$ let \[f_c\prs{a_n} = \sum_{n \in \mbb{N}_+} a_n c_n \text{.}\] Then $c \mapsto f_c$ is an isometry and $\ell_q \cong \prs{\ell_p}^*$.
\end{example}

\begin{example}
The same construction gives and embedding $\ell_1 \rmono \prs{\ell_\infty}^*$, which is linear, norm-preserving and 1-1, but \emph{not} surjective.

For $p < \infty$ it holds that
\[f\prs{\prs{a_n}_{n \in \mbb{N}_+}} = f\prs{\sum_{n \in \mbb{N}_+} a_n e_n} = \sum_{n \in \mbb{N}_+} a_n f\prs{a_n}\]
where $e_n$ is the $n$\textsuperscript{th} basis vector.
But, in $\ell^\infty$ it doesn't hold that $\prs{a_n}_{n \in \mbb{N}_+} \neq \sum_{n \in \mbb{N}_+} a_n e_n$.
For example,
\[\norm{\prs{1,1,1, \ldots} - \sum_{n \in [N]} e_n}_{\infty} = \norm{\prs{0,0,\ldots, 0, 1, 1, 1, \ldots}} = 1 \not\rightarrow 0 \text{.}\]
But, we do have $c_0^* \cong \ell_1$ with the same proof as for $\ell_p$.
\end{example}

\begin{example}
Let $X$ be a compact metric space. We want to understand $\mcal{C}\prs{X}^*$. If $\mu$ is a finite Borel measure on $X$, we can define
\begin{align*}
\Phi_\mu \colon \mcal{C}\prs{X} &\to \mbb{C} \\
f &\mapsto \int_X f \diff \mu \text{.}
\end{align*}
This is a linear functional.
It holds that
\[\abs{\Phi_\mu\prs{f}} = \abs{\int_X f \diff \mu} \leq \int_X \abs{f} \diff \mu \leq \norm{f}_\infty \mu\prs{X} \text{.}\]
So, $\Phi_\mu \in \mcal{C}\prs{X}^*$ and $\norm{\Phi_\mu}^* \leq \mu\prs{X}$. Taking $f \equiv 1$ we see actually that $\norm{\Phi_\mu}^* = \mu\prs{X}$.

If $\mu,\nu$ are two measure on $X$, let
\[\Phi\prs{f} = \int_X f \diff \mu -i \int_X f \diff \nu \in \mcal{C}\prs{X}^* = \int_X f \diff \prs{\mu - i\nu} \text{.}\]
Then $\Phi \in \mcal{C}\prs{X}^*$.
\end{example}

\begin{definition}[Complex Measure]
Let $\prs{X,\Sigma}$ be a measurable space. A \emph{complex measure} $\mu$ on $\prs{X, \Sigma}$ is a map $\mu \colon \Sigma \to \mbb{C}$ such that if $\prs{A_n}_{n \in \mbb{N}_+} \subseteq \Sigma$ are pairwise disjoint it holds that
\[\mu\prs{\bigsqcup_{n \in \mbb{N}_+ A_n}} = \sum_{n \in \mbb{N}_+} \mu\prs{A_n} \text{.}\]
\end{definition}

\begin{fact}
\begin{enumerate}
\item If $\mu$ is a complex measure, $\Re \mu, \Im \mu$ are real signed measures.
\item For every signed measure on a space $X$, there is a decomposition $X = P \sqcup N$ such that $\mu\prs{A} \geq 0$ for every $A \in \Sigma$ such that $A \subseteq P$, and $\mu\prs{A} \leq 0$ for every $A \in \Sigma$ such that $A \subseteq N$.\\
This is called the Hahn decomposition of a signed measure.
\item If $\mu$ is a complex measure on a space $X$, we can write
\[\mu = \mu_1 - \mu_2 + i\prs{\mu_3 - \mu_4}\]
where $\prs{\mu_i}_{i \in [4]}$ are (real, non-negative) measures on $X$.
Then
\[\int_X f \diff \mu = \int_X f \diff \mu_1 - \int_X f \diff \mu_2 + i \prs{\int_X f \diff \mu_3 - \int_X f \diff \mu_4} \text{.}\]
\end{enumerate}
\end{fact}

\begin{definition}[Norm of a Measure]
Given a complex norm $\mu$, we define
\[\norm{\mu} \ceq \sup\set{\sum_{i \in \mbb{N}_+} \abs{\mu\prs{A_i}}}{X = \bigsqcup_{i \in [n]} A_i} \text{.}\]
\end{definition}

\begin{theorem}
Every $\Phi \in \mcal{C}\prs{X}^*$ is of the form $\Phi\prs{f} = \int_X f \diff \mu$ for some complex measure $\mu$. Also
$\norm{\Phi} = \norm{\mu}$.
\end{theorem}

\subsection{Applications of Duality \& Some Measure Theory}

\begin{theorem}[Radon Nikodym]
Let $\prs{X, \Sigma, \mu}$ be a finite measure space. Let $\nu$ be be another measure on $\prs{X, \Sigma}$ such that $\nu \ll \mu$. Then $\exists g \geq 0$ such that
\[\nu\prs{A} = \int_A g \diff \mu\]
for all $A \in \Sigma$.
\end{theorem}

\begin{proof}
Let $H \ceq L_2\prs{X, \mu+\nu}$ and let
\begin{align*}
\ell \colon H &\to \mbb{R} \\
f &\mapsto \int_X f \diff \mu \text{.}
\end{align*}
Then
\begin{align*}
\abs{\ell\prs{f}} &=
\abs{\int_X f \diff \mu}
\\&\leq
\prs{\int_X f^2 \diff \mu}^{\frac{1}{2}} \cdot \prs{\int_X 1^2 \diff \mu}^{\frac{1}{2}}
\\&\leq
\mu\prs{X}^{\frac{1}{2}} \cdot \prs{\int_X f^2 \diff \prs{\mu + \nu}}^{\frac{1}{2}}
\\&=
\mu\prs{X} \norm{f}_{L_2\prs{\mu + \nu}} \text{.}
\end{align*}
By Riesz, there's $h \in H$ such that
\[\int f \diff \mu = \ell\prs{f} = \trs{f, h} = \int_X f h \diff \prs{\mu + \nu} \text{.}\]
Then \[\int_X f\prs{-h} \diff \mu = \int f h \diff \nu \text{.}\]
One can check
$0 \leq h \leq 1$. Define $g = \frac{1-h}{h}$, then
\[\int_X fg \diff \mu = \int_X \frac{f}{h} \prs{1-h} \diff \mu = \int_X \frac{f}{h} \cdot h \diff \nu = \int_X f \diff \nu \text{.}\]
Take $f = \chi_A$, from which
\[\int_A g \diff \mu = \int_A 1 \diff \nu = \nu\prs{A} \text{.}\]
\end{proof}

\begin{remark}
The theorem is also true if $\nu$ is a complex measure, with $g \colon X \to \mbb{C}$ in $L_1\prs{\mu}$. We get this by writing $\nu = \prs{\nu_1 - \nu_2} + i\prs{\nu_3 - \nu_4}$.
\end{remark}

\begin{theorem}
Let $\prs{X, \Sigma, \mu}$ a finite measure space. Fix $1 \leq p < \infty$. Then \[\prs{L_p\prs{X, \Sigma, \mu}}^* = L_q\prs{X, \Sigma, \mu}\]
where $\frac{1}{p} + \frac{1}{q} = 1$.
\end{theorem}

\begin{proof}
We work over $\mbb{C}$, and assume $p > 1$.
For every $g \in L_q\prs{X, \Sigma, \mu}$ define $\Phi_g \colon L_p \to \mbb{C}$ by
\[\Phi_g\prs{f} = \int_X f g \diff \mu \text{.}\]
Obviously, $\Phi_g$ is linear.
By Hölder we know
\[\abs{\Phi_g\prs{f}} \leq \norm{f}_p \norm{g}_q \text{,}\]
so $\Phi_g \in L_p^*$ and $\norm{\Phi_g} \leq \norm{g}_q$.
Notice also that the map $g \mapsto \Phi_g$ is linear.

Fix $\Phi \in L_p\prs{\mu}^*$, we should prove $\Phi = \Phi_g$ for some $g \in L_q$.
Define a new measure $\nu$ on $X$ by
\[\nu\prs{A} = \Phi\prs{\chi_A} \text{.}\]
Here, $\chi_A \in L_p\prs{\mu}$ since $\mu$ is a finite measure.
We prove $\nu$ is indeed a (complex) measure.
Fix $\prs{A_n}_{n=1}^{\infty}$ such that $A = \sqcup_{n=1}^{\infty} A_n$. Note that $\sum_{n=1}^{\infty} \chi_{A_n} = \chi_A$ in $L_p$.
Indeed,
\begin{align*}
\norm{\chi_A - \sum_{n=1}^N \chi_{A_n}}_p &=
\norm{\chi_{\bigsqcup_{n=N+1}^\infty A_n}}_p
\\&= \mu\prs{\bigsqcup_{n=N+1}^\infty A_n}^{\frac{1}{p}}
\\&= \prs{\sum_{n=N+1}^\infty \mu\prs{A_n}}^{\frac{1}{p}}
\\&\xrightarrow{N \to \infty} 0 \text{.}
\end{align*}
Now, $\nu$ is indeed a measure since
\begin{align*}
\nu\prs{A} &= \Phi\prs{\chi_A}
\\&=
\Phi\prs{\sum_{n=1}^{\infty} \chi_{A_n}}
\\&= \sum_{n=1}^{\infty} \Phi\prs{\chi_{A_n}}
\\&= \sum_{n=1}^\infty \nu\prs{A_n} \text{.}
\end{align*}

Now, if $\mu\prs{A} = 0$ then $\chi_A = 0$ in $L_p\prs{\mu}$. So
\[\nu\prs{A} < \Phi\prs{\chi_A} = \Phi\prs{0} = 0 \text{,}\]
so by Radon-Nikodym there exists $g \in L_1\prs{\mu}$ such that
\[\Phi\prs{\chi_A} = \nu\prs{A} = \int_A g \diff \mu = \int \chi_A g \diff \mu = \Phi_g\prs{\chi_A} \text{.}\]

We got $\Phi_g$ such that $\Phi = \Phi_g$ on indicators.
If $f = \sum_{i=1}^n \alpha_i \chi_{A_i}$ is a simple function, we have
\begin{align*}
\Phi\prs{f} &= \sum_{i=1}^n \alpha_i \Phi\prs{\chi_{A_i}}
\\&= \sum_{i=1}^n \alpha_i \Phi_g\prs{\chi_{A_i}}
\\&= \Phi_g\prs{f} \text{.}
\end{align*}

Assume now that $f \in L_p$ is bounded with $\abs{f} \leq M$.
Choose simple functions $\prs{f_n}_{n \in \mbb{N}}$ such that $f_n \to f$ almost-everywhere and that $\abs{f_n} \leq M$.
Now
\begin{align*}
\Phi\prs{f} &\underset{\text{DCT}}{=} \lim_{n \to \infty} \Phi\prs{f_n}
\\&=
\lim_{n \to \infty} \int f_n g \diff \mu
\\&\underset{\text{DCT}}{=}
\int \lim_{n \to \infty} f_n g \diff \mu
\\&=
\int fg \diff \mu
\\&=
\Phi_g\prs{f} \text{.}
\end{align*}

Define
\[A_n \ceq \set{x}{\abs{g\prs{x}} \leq n}\]
and set
\[\rho\prs{z} \ceq \fcases{\frac{\bar{z}}{\abs{z}} & z \neq 0 \\ 0 & z= 0}\]
so that $z \rho\prs{z} = \abs{z}$.
Define $f_n \ceq \chi_{A_n} \rho\prs{g} \abs{g}^{q-1}$ which is bounded.

For all $n \in \mbb{N}$ we have
\begin{align*}
\norm{\Phi} \prs{\int_{A_n} \abs{g}^{q} \diff \mu}^{\frac{1}{p}}
&=
\norm{\Phi} \prs{\int_{A_n} \abs{g}^{p\prs{q-1}} \diff \mu}^{\frac{1}{p}}
\\&=
\norm{\Phi} \norm{f_n}_p
\\&\geq
\Phi \prs{f_n}
\\&=
\Phi_g\prs{f_n}
\\&=
\int g f_n \diff \mu
\\&=
\int_{A_n} g \rho\prs{g} \abs{g}^{q-1} \diff \mu
\\&=
\int_{A_n} \abs{g}^q \diff \mu \text{.}
\end{align*}
Hence
\[\prs{\int_{A_n} \abs{g}^q \diff \mu}^{\frac{1}{q}} \leq \norm{\Phi} \text{.}\]
Then by MCT we have
\[\norm{g}_q \leq \norm{\Phi} < \infty \text{.}\]

Hence $g \in L_q\prs{\mu}$ so $\norm{\Phi_g} \leq \norm{g}_q < \infty$ so $\Phi_g, \Phi$ are continuous. Since they're equal on simple functions, they're equal. Also $\norm{\Phi_g} = \norm{\Phi} \geq \norm{g}_q$ where we've just shown the inequality. Hence $\norm{\Phi_g} = \norm{g}_q$.
\end{proof}

\begin{exercise}
Extend the above theorem to $\sigma$-finite spaces.
\end{exercise}

\chapter{Linear Functionals}

\section{Bounded Linear Functionals}

\subsection{The Hahn-Banach Theorem}

We said that $\prs{\ell_p}^* \cong \ell_q$ for $1 \leq p < \infty$, but \emph{not} for $\ell_\infty$.
To show this, we need to find some linear functional $f \in \prs{\ell_\infty}^*$ not of the form $f\prs{a_n} = \sum_{n=1}^{\infty} a_n b_n$.

\begin{definition}[Sublinear Functions]
LEt $X \in \catname{Vect}_{\mbb{R}}$. A function $p \colon X \to \mbb{R}$ is called \emph{sublinear} if the following holds for all $x,y, \in X$ and $\lambda \in \mbb{R}_{\geq 0}$.
\begin{enumerate}
\item $p\prs{x+y} \leq p\prs{x} + p\prs{y}$.
\item $p\prs{\lambda x} = \lambda p\prs{x}$.
\end{enumerate}
\end{definition}

\begin{theorem}[Hahn-Banach, \#1] \label{theorem:hahn_banach}
Let $X \in \catname{Vect}_{\mbb{R}}$ and $p \colon X \to \mbb{R}$ sublinear. Assume $Y \subseteq X$ is a subspace, $\ell \colon Y \to \mbb{R}$ is linear and $\ell \leq p$ on $Y$. Then $\exists \tilde{\ell} \colon X \to \mbb{R}$ linear such that $\tilde{\ell} \leq p$ and $\left. \tilde{\ell} \right|_Y = \ell$.
\end{theorem}

\begin{lemma}\label{lemma:hahn_banach_lemma}
Theorem \ref{theorem:hahn_banach} holds if $\codim Y = 1$.
\end{lemma}

\begin{proof}
Every $x \in X$ is of the form $x = y + \lambda x_0$. We want to define
\[\tilde{\ell}\prs{x} = \tilde{\ell} \prs{y} + \lambda \tilde{\ell}\prs{x_0} = \ell\prs{y} + \lambda \tilde{\ell}\prs{x_0} \text{.}\]
We only get to choose $a = \tilde{\ell}\prs{x_0}$. The goal is to choose $a$ to have
\[\forall y\ in Y \forall \lambda \in \mbb{R} \colon \ell\prs{y} + \lambda a = \tilde{\ell} \prs{y + \lambda x_0} \leq p\prs{y + \lambda x_0} \text{.}\]

For $\lambda > 0$ we get the requirement
\[a \leq \frac{1}{\lambda} \prs{p \prs{y + \lambda x_0} - \ell\prs{y}} = p\prs{\frac{y}{\lambda} + x_0} - \ell\prs{\frac{y}{\lambda}} \text{.}\]
For $\lambda < 0$, write $\lambda = - \mu$ with $\mu > 0$. Denote also $z \ceq y$. We get the requirement
\[\ell\prs{z} - \mu a \leq p\prs{z - \mu x_0}\]
so
\[a \geq \frac{1}{\mu} \prs{\ell\prs{z} - p\prs{z-\mu x_0}} = \ell\prs{\frac{z}{\mu}} - p\prs{\frac{z}{\mu} - x_0} \text{.}\]

Choosing such $a$ is possible iff
\[\sup_{\substack{Z \in Y \\ \mu > 0}} \brs{\ell\prs{\frac{z}{\mu}} - p\prs{\frac{z}{\mu} - x_0}} \leq \inf_{\substack{y \in Y \\ \lambda \geq 0}} \brs{p\prs{\frac{y}{\lambda} + x_0} - \ell\prs{\frac{y}{\lambda}}} \text{.}\]
We prove this is the case.

Let $y,z \in Y$ and $\lambda,\mu \in \mbb{R}_+$.
We indeed have
\begin{align*}
\ell\prs{\frac{z}{\mu}} + \ell\prs{\frac{y}{\lambda}} &= \lambda\prs{\frac{z}{\mu} + \frac{y}{\lambda}}
\\&\leq
p\prs{\frac{z}{\mu} - x_0 + \frac{y}{\lambda} + x_0}
\\&\leq
p\prs{\frac{z}{\mu} - x_0} + p\prs{\frac{y}{\lambda} + x_0} \text{.}
\end{align*}
\end{proof}

\begin{proof}[\ref{theorem:hahn_banach}]
Let
\[\mcal{F} \ceq \set{\prs{Z, f}}{\substack{Y \subseteq Z \subseteq X \\ f \colon Z \to \mbb{R} \\ \left. f \right|_Y = \ell \\ f \leq p}} \text{.}\]

We have $\prs{Y, \ell} \in S$ so $S \neq \ns$.
Say $\prs{Z_1, f_1} \leq \prs{Z_2, f_2}$ if $Z_1 \subseteq Z_2$ and $\left. f_2 \right|_{Z_1} = f_1$.

Take a chain $\set{\prs{Z_i, f_i}}_{i \in I}$ in $\mcal{F}$. We define $Z_\infty = \bigcup_{i \in I} Z_i$ and $f_\infty\prs{x} = f_i\prs{x}$ if $x \in Z_i$. Check that $\prs{Z_\infty, f_\infty} \in S$ and this is an upper bound to the chain.
By Zorn's lemma, there's a maximal element $\prs{Z_0, f_0}$ of $\mcal{F}$.
By \ref{lemma:hahn_banach_lemma} we get $Z_0 = X$, hence we're done.
\end{proof}

\begin{theorem}[Hahn Banach \#2]
\label{theorem:hahn_banach_2}
Let $\prs{X, \norm{\cdot}}$ be a normed space over $\mbb{F}$, let $Y \subseteq X$ be a subspace and let $f \in Y^*$. Then $\exists \tilde{f} \in X^*$ such that $\rest{\tilde{f}}{Y} = f$ and $\norm{\tilde{f}}_{X^*} = \norm{f}_{Y^*}$.   
\end{theorem}

\begin{proof}
\begin{itemize}
\item Assume first $\mbb{F} = \mbb{R}$.
Just take $\ell = f$ and $p\prs{x} = \norm{f}_{Y^*} \norm{x}$.
Then for all $y \in Y$ we have
\[f\prs{y} = \ell\prs{y} \leq p\prs{y} = \norm{f} \norm{y} \text{.}\]
Hence by \ref{theorem:hahn_banach} there's $\tilde{f}$ such that $\rest{\tilde{f}}{Y} = f$ and
\[\forall x \in X \colon \tilde{f}\prs{x} \leq \norm{f}_{Y^*} \norm{x} \text{.}\]
Taking $-x$ instead of $x$ we get
\[-\tilde{f}\prs{x} \leq \norm{f}_{Y^*} \norm{x} \text{.}\]
Hence
$\norm{\tilde{f}}{X^*} \leq \norm{f}_{Y^*}$.
\item Assume now that $\mbb{F} = \mbb{C}$.
We can think of $\prs{X, \norm{\cdot}}$ also as a space over $\mbb{R}$. Write
\[f\prs{y} = g\prs{y} + i h\prs{y}\]
where $g,h \colon Y \to \mbb{R}$ are $\mbb{R}$-linear.
We have
\[g\prs{iy} + ih\prs{iy} = f\prs{iy} = if\prs{y} = -h\prs{y} + i g\prs{y}\]
so
\[h\prs{y} - g\prs{iy} \text{.}\]
Hence
\[f\prs{y} = g\prs{y} - ig\prs{iy} \text{.}\]
Obviously,
\[\abs{g\prs{y}} \leq \abs{f\prs{y}} \text{,}\]
so
\[\norm{g}_{Y^*} \leq \norm{f}_{Y^*} \text{.}\]
By the real case, there's $\tilde{g} \colon X \to \mbb{R}$ which is $\mbb{R}$-linear such that $\rest{\tilde{g}} = g$ and \[\norm{\tilde{g}}_{X^*} \leq \norm{g}_{Y^*} \leq \norm{f}_{Y^*} \text{.}\]

Define
\[\tilde{f}\prs{x} = \tilde{g}\prs{x} - i \tilde{g}\prs{ix} \text{.}\]
We get that $\tilde{f}\prs{x}$ is $\mbb{R}$-linear, but also
\[i \tilde{f}\prs{x} = i \tilde{g}\prs{x} + \tilde{g}\prs{ix} = \tilde{f}\prs{ix}\]
so $\tilde{f}$ is $\mbb{C}$-linear.
Also
$\rest{\tilde{f}}{Y} = f$, so we're left to check the norm of $\tilde{f}$.
Fix $x \in X$ and write $\tilde{f}\prs{x} = re^{i\theta}$.
Now
\begin{align*}
\abs{\tilde{f}\prs{x}} &= r
\\&=
e^{-i\theta} \tilde{f}\prs{x}
\\&=
\tilde{f}\prs{e^{-i\theta x}}
\\&=
\tilde{g} \prs{e^{-i\theta x}}
\\&\leq
\norm{\tilde{g}} \norm{e^{-i\theta} x}
\\&=
\norm{\tilde{g}}\norm{x} \text{.}
\end{align*}
So,
\[\norm{\tilde{f}}_{X^*} \leq \norm{\tilde{g}}_{X^*} \leq \norm{f}_{Y^*} \text{.}\]
\end{itemize}
\end{proof}

\begin{corollary} \label{corollary:hahn_banach_1}
For every $x_0 \in X \setminus \set{0}$ there's $f \in X^*$ such that $\norm{f} = 1$ and $f\prs{x_0} = \norm{x_0}$.
\end{corollary}

\begin{proof}
Let $Y= \spn\set{x_0}$ and
\begin{align*}
f \colon Y &\to \mbb{F} \\
\lambda x_0 &\mapsto \lambda \norm{x_0} \text{.}
\end{align*}
Then $\norm{f}_{Y^*} = 1$ and $f\prs{x_0} = \norm{x_0}$. Now extend $f$ to all of $X$.
\end{proof}

\begin{corollary}
If $f\prs{x_1} = f\prs{x_2}$ for all $f \in X^*$, then $x_1 = x_2$.
\end{corollary}

\begin{proof}
If $x_1 \neq x_2$, by \ref{corollary:hahn_banach_1} there's $f \in X^*$ with $\norm{f} = 1$ such that $f\prs{x_1 - x_2} = \norm{x_1 - x_2} \neq 0$. Then $f\prs{x_1} \neq f\prs{x_2}$.
\end{proof}

\begin{corollary} \label{corollary:hahn_banach_2}
Let $E \leq X$ and let $x_0$ such that
\[d \ceq d\prs{x_0, E} > 0 \text{.}\]
There's $f \in X^*$ with $\norm{f} = 1$ such that $f\prs{E} = 0$ and $f\prs{x_0} = d$.
\end{corollary}

\begin{proof}
Take $Y = \spn\set{E, x_0}$ and let
\begin{align*}
f \colon Y &\to \mbb{F} \\
e + \lambda x_0 &\mapsto \lambda \cdot d \text{.}
\end{align*}
Then
\begin{align*}
\norm{e + \lambda x_0} &= \abs{\lambda} \norm{x_0 - \prs{- \frac{1}{\lambda}}}
\\&\geq \abs{\lambda} d
\\&= \abs{f\prs{e + \lambda x_0}} \text{.}
\end{align*}
So $\norm{f} \leq 1$, and one can check that actually $\norm{f} = 1$.

By \ref{theorem:hahn_banach_2} we can extend $f$ to $X$, which finishes the proof.
\end{proof}

\begin{corollary} \label{corollary:hahn_banach_3}
Let $A \subseteq X$ and $x \in X$. The following are equivalent.
\begin{enumerate}
\item $x \in \overline{\spn\prs{A}}$.
\item For every $f \in X^*$ such that $f\prs{A}$ it holds that $f\prs{x} = 0$.
\end{enumerate}
\end{corollary}

\begin{proof}
1 $\implies$ 2 is obvious.
\\
For the other direction, let $E \ceq \overline{\spn\prs{A}}$. If $x \notin E$ then $d\prs{x, E} = d > 0$. From \ref{corollary:hahn_banach_2} there's $f \in X^*$ with $\norm{f} = 1$ such that $f\prs{E} = 0$ and $f\prs{x} = d \neq 0$.
\end{proof}

\subsection{Application of Hahn-Banach - Banach Limits}

\begin{definition}[Banach Limit]
A \emph{Banach limit} is a linear functional $f \colon \ell_\infty \to \mbb{F}$ such that the following holds.
\begin{enumerate}
\item If $\prs{a_n}_{n \in \mbb{N}_+}$ is such that $a_n \geq 0$ for all $n \in \mbb{N}_+$, then $f\prs{\prs{a_n}_{n \in \mbb{N}_+}} \geq 0$.
\item $f\prs{\prs{a_{n+1}}_{n \in \mbb{N}_+}} = f\prs{\prs{a_n}_{n \in \mbb{N}_+}}$.
\item If $a_n \xrightarrow{n\to\infty} L$ then $f\prs{\prs{a_n}_{n \in \mbb{N}_+}} = L$.
\end{enumerate}
\end{definition}

\begin{remark}
It holds that
\begin{align*}
f\prs{\prs{a_n}_{n \in \mbb{N}_+}} &\leq f\prs{\prs{\norm{\prs{a_n}_{n \in \mbb{N}_+}}_\infty, \norm{\prs{a_n}_{n \in \mbb{N}_+}}_\infty, \norm{\prs{a_n}_{n \in \mbb{N}_+}}_\infty, \ldots}} = \norm{\prs{a_n}_{n \in \mbb{N}_+}}_\infty \text{.}
\end{align*}
Similarly
\[-f\prs{\prs{a_n}_{n \in \mbb{N}_+}} \leq \norm{\prs{a_n}_{n \in \mbb{N}_+}}_\infty \text{.}\]
From this it follows that $\norm{f} \leq 1$, and by looking at a constant sequence we actually get $\norm{f} = 1$.
\end{remark}

\begin{remark}
A Banach limit $f$ is in $\prs{\ell_\infty}^*$ and this is \emph{not} of the form $f\prs{\prs{a_n}_{n \in \mbb{N}_+}} = \sum_{n \in \mbb{N}_+} a_n b_n$. We show the latter part.

Assume $f\prs{\prs{a_n}_{n \in \mbb{N}_+}} = \sum_{n=1}^{\infty}$. Then
\begin{align*}
0 &= f\prs{e_k} \\&= \sum_{n \in \mbb{N}_+} \prs{e_k}_n b_n \\&= \sum_{n \in \mbb{N}_+} \delta_{k,n} b_n \\&= b_k
\end{align*}
so $f = 0$, which is a contradiction to $f\prs{1,1,1,\ldots} = 1$.
\end{remark}

\begin{theorem}
There exists a Banach limit.
\end{theorem}

\begin{remark}
Without parts (1,2) of the definition, this is obvious from \ref{theorem:hahn_banach_2}. One can extend $f\prs{\prs{a_n}_{n \in \mbb{N}_+}} = \lim_{n \to \infty} a_n$ from $c$ to $\ell_\infty$.
\end{remark}

\begin{proof}
Let
\[E \ceq \set{\prs{a_{n+1} - a_n}_{n \in \mbb{N}_+}}{\prs{a_n}_{n \in \mbb{N}_+} \in \ell_\infty}\]
and $x_0 = \prs{1, 1, 1, \ldots}$ the constant sequence $1$. We claim
$d\prs{x_0, E} = 1$.
Let $b \ceq \prs{b_n}_{n \in \mbb{N}_+} \in E$ and let $d \ceq \norm{x_0 - b}$. Then
\begin{align*}
nd &\geq \sum_{k \in [n]} \abs{\prs{x_0}_k - b_k}
\\&\geq
\sum_{k \in [n]} \prs{1 - b_k}
\\&=
n - \sum_{k \in [n]} b_k \text{.}
\end{align*}
We claim the last sum is bounded. Indeed, if $b_k = a_{k+1} - a_k$ then
\[\sum_{k \in [n]} b_k = a_{n+1} - a_1 \text{.}\]
Hence, dividing by $n$ we get
\[d \geq 1 - \frac{1}{n} \sum_{k \in [n]} a_k \xrightarrow{n \to \infty} 1 \text{.}\]
Hence $d\prs{x_0, E} \geq 1$, but obviously $d\prs{x_0, E} \leq \norm{x_0} = 1$.

By \ref{corollary:hahn_banach_2} there's $f \in \prs{\ell_\infty}^*$ such that $f\prs{E} = 0$, $f\prs{x_0} = 1$ and $\norm{f} = 1$.
Hence we have property 2 of the Banach limit. To get property 1 fix a sequence $a = \prs{a_n}_{n \in \mbb{N}_+}$ such that $a_n \geq 0$ and assume without loss of generality that $\norm{a}_{\infty} = 1$.
Then
\begin{align*}
f\prs{a} &= f\prs{x_0 - \prs{x_0 - a}} \\&= f\prs{x_0} - f\prs{x_0 - a} \geq 1 - \norm{x_0 - a} \\&\geq 1 - 1 \\&= 0
\end{align*}
where in the last inequality we use $a_n \in \brs{0,1}$ for all $n \in \mbb{N}_+$. This proves property 1.

Note that
\begin{align*}
\forall a \in \ell_\infty \forall k \inf_{mbb{N}} f\prs{\prs{a_n}_{n \in \mbb{N}_+}} = f\prs{\prs{a_{n+k}}_{n \in \mbb{N}_+}} \leq \sup_{n \geq k} a_n \text{.}
\end{align*}
Hence
\[f\prs{\prs{a_n}_{n \in \mbb{N}_+}} \leq \inf_{k \in \mbb{N}} \sup_{n \geq k} a_n = \limsup_{n \to \infty} a_n \text{.}\]
Taking $-a$ instead of $a$ we get
\[-f\prs{a} \leq \limsup_{n \to \infty}\prs{-a} = -\liminf_{n \to \infty} a_n\]
so
\[f\prs{a} \geq \liminf_{n\to\infty} a_n \text{.}\]
Hence if $a_n \to L$ also $f\prs{a} = L$.
\end{proof}

\subsection{Application of Hahn-Banach - Reflexive Spaces}

\begin{definition}
For every $x \in X$ define $\mrm{ev}_x \in X^{**} = \prs{X^*}^*$ by
\[\forall f \in X^* \colon \mrm{ev}_x\prs{f} = f\prs{x} \text{.}\]
\end{definition}

\begin{proposition}
The map $x \mapsto \mrm{ev}_x$ is a linear norm-preserving map $X \to X^{**}$.
\end{proposition}

\begin{proof}
Linearity is easy. Note that
\[\abs{\mrm{ev}_x\prs{f}} = \abs{f\prs{x}} \leq \norm{f} \norm{x} = \norm{x} \norm{f} \text{,}\]
so by definition $\norm{\mrm{ev}_x} \leq \norm{x}$. By Hahn-Banach there's $f$ with $\norm{f} = 1$ and $f\prs{x} = \norm{x}$. Then \[\norm{x} = \abs{f\prs{x}} = \abs{\mrm{ev}_x\prs{f}} \leq \norm{\mrm{ev}_x}\norm{f} = \norm{\mrm{ev}_x} \text{.}\]
Hence $x \mapsto \mrm{ev}_x$ is norm-preserving.
\end{proof}

\begin{definition}
A normed space $X$ is called \emph{reflexive} if $X^{**}$ is of the form $\mrm{ev}_x$ for some $x \in X$.
\end{definition}

\begin{example}
If $p \in \prs{1, \infty}$, the spaces $\ell_p, L_p$ are reflexive since for $\frac{1}{p} + \frac{1}{q} = 1$ we know $\prs{L_p}^* \cong L_q$ and $\prs{L_q}^* \cong L_p$, and similarly for $\ell_p$.

However, $\ell_1$ is not reflexive since $\prs{\ell_1}^* \cong \ell_\infty$ and $\prs{\ell_\infty}^* \not\cong \ell_1$.
Similarly, $c_0$ is not reflexive since $c_0^* \cong \ell_1$ but $\ell_1^* \cong \ell_\infty \not\cong c_0$.
\end{example}

\subsection{Application of Hahn-Banach - Approximation Theory}

\begin{example}
The space $\spn\set{x^k}{k \in \mbb{Z}_{\geq 0}}$ is dense in $\mcal{C}\prs{\brs{0,1}}$ by the Weistrass approximation theorem.
\end{example}

\begin{example}
Consider \[\mcal{C}\prs{\mbb{T}} \ceq \set{f \colon R \to \mbb{C}}{\text{$f$ is continuous and $2 \pi$-periodic}} \text{.}\]
Then $\spn\set{e^{2 pi i n}}{n \in \mbb{Z}_{\geq 0}}$ is dense in $\mcal{C}\prs{\mbb{T}}$ by the second Weierstrass approximation theorem / Fejer's theorem.
\end{example}

\begin{theorem}
Fix a weight function $w \colon \mbb{R} \to \mbb{R}$ such that $0 < w\prs{t} < A e^{-B \abs{t}}$.
Then
\[E \ceq \spn\set{t^n w}{n \in \mbb{Z}_{\geq 0}}\]
is dense in $c_0\prs{\mbb{R}}$.
\end{theorem}

\begin{proof}
\begin{description}
\item[Step 1:]

We show that $c^{i \alpha t}w \in \bar{E}$ for every $\alpha \in \mbb{R}$.
Let $f \in c_0\prs{\mbb{R}}^*$ such that $f\prs{E} = 0$. For $z \in \mbb{C}$ write $\rho_z\prs{t} \ceq e^{i z t} w$.
It holds that
\begin{align*}
\abs{\rho_z\prs{t}} &= \abs{e^{i z t}}w
\\&=
e^{\Re\prs{i z t}} w
\\&\leq
e^{-\prs{\Im z}t} A^{- B \abs{t}}
\\&\leq
A e^{\prs{\abs{\Im z} - B}\abs{t}} \text{.}
\end{align*}
If $z \in S \ceq \set{z}{\abs{\Im z} < B}$ then $\rho_z \in c_0\prs{\mbb{R}}$. Define $\phi \colon S \to \mbb{C}$ by $\phi\prs{t} = f\prs{\rho_t}$.

Note that
\begin{align*}
\prs{\frac{\rho_{z+h} - \rho_z}{h}}\prs{t} &= \frac{e^{i\prs{z+h} t} w\prs{t} - e^{izt} w\prs{t}}{h}
\\&=
w\prs{t} e^{i z t} \frac{e^{iht} - 1}{h}
\\&=
w\prs{t} e^{i z t} it \cdot \frac{e^{iht} - 1}{ith}
\end{align*}
where the last expression converges uniformly to $e^{izt}$ as $h \to 0$ since $w\prs{t} e^{i z t} it$ is bounded and since $\frac{e^{iht} - 1}{ith}$ converges uniformly to $1$.

Hence
\begin{align*}
\phi'\prs{z}
&=
\lim_{h \to 0} \frac{\phi\prs{z+h} - \phi\prs{z}}{h}
\\&=
\lim_{h \to 0} f\prs{\frac{\rho_{z + h} - \rho_z}{h}}
\\&=
f \prs{\lim_{h \to 0} \frac{\rho_{z+h} - \rho_h}{h}}
\\&=
f\prs{it e^{izt} w\prs{t}} \text{.}
\end{align*}
In particular, $\phi$ is analytic and $\phi'\prs{0} = f\prs{f\prs{it w\prs{t}}} = 0$ where the latter equality is by the choice of $f$.

Similarly 
\[\phi^{\prs{n}}\prs{t} = f\prs{\prs{it}^n e^{izt} w\prs{t}}\]
so
\[\phi^{\prs{n}}\prs{0} = f\prs{i^n t^n w\prs{t}} = 0 \text{.}\]
Hence by analyticity and complex analysis, $\phi \equiv 0$. Hence $f\prs{\rho_t} = 0$ for every $z \in S$ and every $f \in c_0\prs{\mbb{R}}^*$ for which $f\prs{E} = 0$.
Hence $\rho_z \in \bar{E}$ for every $z \in S$.

\item[Step 2:]
We'll show that $\mcal{C}_c\prs{\mbb{R}} \subseteq \bar{E}$ which finishes the proof because $\overline{\mcal{C}_c\prs{\mbb{R}}} = c_0\prs{\mbb{R}}$.

Let $u \in \mcal{C}_c\prs{\mbb{R}}$ and let $\prs{-M, M}$ contain $\supp\prs{u}$. Define
\[u\prs{t} \ceq \frac{u\prs{\frac{M}{\pi} t}}{w\prs{\frac{M}{\pi} t}} \text{.}\]
This is supported in $\prs{-\pi, \pi}$.
We can then extend $v$ to $\mcal{C}\prs{\mbb{T}}$. By Weirstrass' approximation there are trigonometric polynomials $\prs{p_n}_{n \in \mbb{N}}$ such that $p_n \xrightarrow{n\to\infty} v$ uniformly.
Then
\[p_n\prs{\frac{\pi}{M} t} w\prs{t} \xrightarrow{n\to\infty} v\prs{\frac{\pi}{M} t} w\prs{t} = u\prs{t}\]
uniformly since $w\prs{t}$ is bounded.
By the first step, we have $p_n\prs{\frac{\pi}{M} t} w\prs{t} \in \bar{E}$ hence also $u \in \bar{E}$, as required.
\end{description}
\end{proof}

\subsection{Application of Hahn-Banach - Convex Separation}

\begin{definition}[The Minkowski Functional]
Let $X \in \catname{Vect}_{\mbb{R}}$ and let $K \subseteq X$ be convex such that $0 \in K$.
Define the \emph{Minkowski function on $X$}
\begin{align*}
p_K \colon X &\to \brs{0, \infty} \\
x &\mapsto \inf\set{\lambda > 0}{\frac{x}{\lambda} \in K} \text{.}
\end{align*}
\end{definition}

\begin{definition}[Internal Point]
Let $X \in \catname{Vect}_{\mbb{R}}$ and $A \subseteq X$. A point $a \in A$ is \emph{internal} if
\[\forall y \in X \exists t_0 \in \mbb{R} \forall t \leq t_0 \colon a + ty \in A \text{.}\]
\end{definition}

\begin{remark}
If $X$ is normed and $A \subseteq X$, every interior point of $A$ is internal but the converse is generally false. E.g. take the union of two tangent circles and the line tangent to both of them. The tangential point is an internal point but not an interior point.
\end{remark}

\begin{fact}
Let $X \in \catname{Vect}_{\mbb{R}}$ and let $K \subseteq X$ be convex such that $0 \in K$ is internal. Then $p_K \colon X \to \left[0, \infty \right)$ is sub-linear.
\end{fact}

\begin{proof}
One checks that $p_K\prs{\lambda x} = \lambda p_K\prs{x}$ for $\lambda > 0$.

We check the triangle inequality.
Let $x,y \in X$. There are $\prs{\lambda_n}_{n \in \mbb{N}}$ and $\prs{\mu_n}_{n \in \mbb{N}}$ such that $\frac{x}{\lambda_n}, \frac{y}{\mu_n} \in K$ for all $n \in \mbb{N}$ and that
\begin{align*}
\lambda_n \xrightarrow{n\to\infty} p_k\prs{x} \\
\mu_n \xrightarrow{n\to\infty} p_k\prs{y} \text{.}
\end{align*}

We have
\begin{align*}
\frac{x+y}{\lambda_n + \mu_n} &=
\frac{\lambda_n}{\lambda_n + \mu_n} \frac{x}{\lambda_n} + \frac{\mu_n}{\lambda_n + \mu_n} \frac{y}{\mu_n} \in K
\end{align*}
by convexity of $K$. So $p_K\prs{x+y}$.
So \[p_K\prs{x+y} \leq \lambda_n + \mu_n \xrightarrow{n\to\infty} p_K\prs{x} + p_K\prs{y}\text{.}\]
\end{proof}

\begin{remark}
We cannot in general reconstruct $K$ from $p_K$. We know
\begin{align*}
p_K^{-1}\prs{\left[0,1\right)} \subseteq K \subseteq p_K^{-1}\prs{\brs{0,1}} \text{,}
\end{align*}
but don't know if a point $p \in X$ for which $p_K\prs{p} = 1$ are in $K$ or not.
\end{remark}

\begin{definition}[Separating Function]
Let $X$ be a real Banach space.
Let $A,B \subseteq X$ be disjoint. A linear functional $f \colon X \to \mbb{R}$ separates $A$ and $B$ if $f \neq 0$ and
\[\sup_{a \in A} f\prs{a} \leq \inf_{b \in B} f\prs{b} \text{.}\]
\end{definition}

\begin{theorem}
Let $X$ be a real vector space and $A,B \subseteq X$ be convex and disjoint such that $A$ has an internal point. Then there exists $f \colon X \to \mbb{R}$ which separates $A,B$.
\end{theorem}

\begin{proof}
\begin{description}
\item[Step 1:]
Assume first that $B = \set{x_0}$ and that $0 \in A$ is internal.
Let $Y = \spn\set{x_0}$ and define
\begin{align*}
\ell \colon Y &\to \mbb{R} \\
\lambda x_0 &\mapsto \lambda p_A\prs{x_0} \text{.}
\end{align*}
Then
\begin{align*}
\forall \lambda >0 \colon \ell\prs{\lambda x_0} &= \lambda p_K\prs{x_0} = p_A\prs{\lambda x_0} \\
\forall \lambda \leq 0 \colon \ell\prs{\lambda x_0} &= \lambda p_A\prs{x_0} \leq 0 \leq p_A\prs{\lambda x_0}
\end{align*}
so $\lambda \leq p$.

We can extend $\ell$ to a function $f \colon X \to \mbb{R}$ such that $\rest{f}{Y} = \ell$ and $f \leq p_A$.
But
\[\sup_{a \in A} f\prs{a} \leq \sup_{a \in A} p_A\prs{a} \leq 1 \leq p_A\prs{x_0} = f\prs{x_0} \text{.}\]

\item[Step 2:]
In general, let $a_0 \in A$ be internal and let $b_0 \in B$. Let
\[C \ceq A \setminus B + b_0 - a_0 = \set{a-b + b_0 - a_0}{a \in A, b \in B} \text{.}\]
This is convex by writing down the definition, and $0 = a_0 - b_0 + b_0 - a_0 \in C$ is an internal point because $a_0$ is internal in $A$.

But, $b_0 - a_0 \notin C$ for otherwise $a=b$ which implies $A \cap B \neq \ns$. Hence, there's $f \colon X \to \mbb{R}$ such that
\[\sup_{\substack{a \in A \\ b \in B}}\prs{f\prs{a} - f\prs{b} + f\prs{b_0} - f\prs{a_0}} = \sup_{c \in C} f\prs{c} \leq f\prs{b_0 - a_0} = f\prs{b_0} - f\prs{a_0}\text{.}\]
So
\[\sup_{\substack{a \in A \\ b \in B}} \prs{f\prs{a} - f\prs{b}} \leq 0\]
which implies
\[\sup_{a \in A} f\prs{a} \leq \inf_{b \in B} f\prs{b} \text{.}\] 
\end{description}
\end{proof}

\begin{example}
Let $X \ceq \mbb{R}\brs{x}$ and let $A$ be the subspace of polynomials with positive leading coefficient. This is convex. Take $B = \set{0}$, so that $A \cap B = \ns$.

Assume there's a separating $f \colon X \to \mbb{R}$ for $A,B$. I.e.
\[\sup_{a \in A} f\prs{a} \leq f\prs{a} = 0 \text{.}\]

Let $p \in X$ and $M > \deg p$. Then
\[\forall \delta >0 \colon \delta x^m + p \in A \text{.}\]
Then
\[\forall \delta > 0 \colon \delta f\prs{x^m} + f\prs{p} = f\prs{\delta x^m + p} \leq 0 \text{.}\]
By sending $\delta \to 0$ we get $f\prs{p} \leq 0$. By considering $f\prs{-p}$ we get $f\prs{p} \geq 0$ so $f \equiv 0$, a contradiction.

Note that $A$ doesn't have an internal point, for otherwise the theorem would imply there's a separation for $A,B$.
\end{example}

We've so far discussed separation when $X$ isn't normed, but there might be a norm on it.

\begin{proposition}
Let $X$ be a real Banach space and let $f \colon X \to \mbb{R}$ be linear. Let $\lambda \in \mbb{R}$ and $H \ceq f^{-1}\prs{\lambda}$. Then
\begin{enumerate}
\item If $f$ is bounded, $H$ is closed.
\item If $f$ is not bounded, $H$ is dense.
\end{enumerate}
\end{proposition}

\begin{proof}
\begin{enumerate}
\item This is immediate from topology.
\item It holds that
\begin{align*}
\sup_{x \in X} \frac{\abs{f\prs{x}}}{\norm{x}} = \infty \text{.}
\end{align*}
Take $\prs{x_n}_{n \in \mbb{N}}$ such that $\frac{\abs{f\prs{x_n}}}{\norm{x_n}} \xrightarrow{n\to\infty} \infty$ and $y_n \ceq \frac{x_n}{f\prs{x_n}}$ such that $f\prs{y_n} = 1$ and $\norm{y_n} \to \infty$.
Then
\begin{align*}
\forall x\in X \colon x = \lim_{n \to \infty} \prs{x - f\prs{x} y_n + \lambda y_n} \text{.}
\end{align*}
But,
\[f\prs{x - f\prs{x} y_n + \lambda y_n} = f\prs{x} - f\prs{x} \cdot 1 + \lambda \cdot 1 = \lambda\]
so
\[x - f\prs{x} y_n + \lambda y_n \in H \text{,}\]
so $x$ is the limit of elements of $H$, which implies $x \in \bar{H}$.
\end{enumerate}
\end{proof}

\begin{corollary}
Let $X$ be a real Banach space and let $A,B \subseteq X$ be convex disjoint sets such that $A$ has an interior point $y_0$.
There's $f \in X^*$ separating $A,B$.
\end{corollary}

\begin{proof}
We know there's some linear functional $f \colon X \to \mbb{R}$, which we don't know is continuous.
We have
\[\sup_{a \in B\prs{y_0, r}} f\prs{a} \leq \sup_{a \in A} f\prs{a} \leq \lambda = \inf_{b \in B} f\prs{b} \text{.}\]
Let
\[H \ceq f^{-1}\prs{\lambda + 1} \text{.}\]
Then $H \cap B\prs{y_0, r} = \ns$ so $H$ is not dense, and by the proposition this implies $f$ is bounded, i.e. continuous.
\end{proof}

\section{Banach Spaces}

\subsection{Baire Category Theorem \& Uniform Boundedness}

\begin{definition}
Let $X$ be a topological space and let $A \subseteq X$.
We say $A$ is \emph{nowhere dense} if $\int \bar{A} = \ns$.
We say $A$ is \emph{meagre} (altenatively, a set of the set cateogry) if $A = \bigcup_{n \in \mbb{N}} A_n$ where each $A_n$ is nowhere dense.
We say $A$ is \emph{comeagre} if $A^C$ is meagre.
\end{definition}

\begin{theorem}[Baire]
A complete metric space is not meagre.
\end{theorem}

\begin{theorem}[Uniform Boundedness Principle]
Let $Y$ be a Banach space and $Y$ a normed space. Let $\mcal{F} \subseteq \mscr{L}\prs{X,Y}$.
Assume
\[\forall x \in X \colon \sup_{T \in \mcal{F}} \norm{Tx} < \infty \text{.}\]
Then
\[\sup_{T \in \mcal{F}} \norm{T} < \infty \text{.}\]
\end{theorem}

This follows immediately from the following theorem and Baire's category theorem.

\begin{theorem}
Let $X$ be a Banach space, $Y$ a normed space and $\mcal{F} \subseteq \mscr{L}\prs{X,Y}$. Assume $\set{x}{\sup_{T \in \mcal{F}} \norm{{T x} < \infty}}$ is not meagre. Then $\sup_{T \in \mcal{F}} < \infty$.
\end{theorem}

\begin{proof}
Let
\[A_n \ceq \set{x}{\sup_{T \in \mcal{F}} \norm{Tx} \leq n} \text{.}\]
Then $A_n$ are closed because if $\phi_T\prs{x} = \norm{Tx}$ we get \[A_n = \bigcap_{T \in \mcal{F}} \phi_T^{-1} \prs{\brs{0,n}} \text{.}\]
Moreover, $\bigcup_{n \in \mbb{N}_+} A_n = A$ so there's $n \in \mbb{N}$ such that $\int \overline{A_n} \neq \ns$. Then there's $y_0 \in X$ and $r > 0$ such that $B\prs{y_0, r} \subseteq A_n$.
If $\abs{z} < r$ we can write
\[\norm{Tz} = \norm{T \prs{\frac{y_0 + Z}{z}} - T \prs{\frac{y_0 - z}{2}}} \leq \frac{1}{2} \prs{\norm{T \prs{y_0 + Z}} + \norm{T \prs{y_0 - z}}} \leq \frac{1}{2} \prs{n+n} = n \text{.}\]
Now
\[\forall T \in \mcal{F} \forall x \in X \colon \norm{Tx} = \norm{T \prs{\frac{r x}{2 \norm{x}}}} \cdot \frac{2 \norm{x}}{r} \leq \frac{2n}{r} \norm{x} \text{.}\]
Hence
\[\sup_{T \in \mcal{F}} \norm{T} \leq \frac{2n}{r} < \infty \text{.}\]
\end{proof}

\begin{corollary}
Let $X$ be a Banach space and $A \subseteq X$. Then $A$ is bounded iff $f\prs{A}$ is bounded for all $f \in X^*$.
\end{corollary}

\begin{proof}
\begin{itemize}
\item Assume $A$ is bounded and $f \in X^*$. We get
\[\abs{f\prs{a}} \leq \norm{f} \norm{a} \leq \prs{\sup_{a \in A} \norm{a}}\norm{f}\]
so $f\prs{A}$ is bounded.

\item Assume $f\prs{A}$ is bounded for all $f \in X^*$. Define
\[\mcal{F} \ceq \set{\mrm{ev}_x}{x \in A} \subseteq X^{**} \subseteq \mscr{L}\prs{X^*, \mbb{F}} \text{.}\]
Then
\begin{align*}
\forall f \in X^* \colon \sup_{T \in \mcal{F}} \norm{Tf} = \sup_{x \in A} \abs{\mrm{ev}_x\prs{f}} = \sup_{x \in A} \abs{f\prs{x}} < \infty \text{.}
\end{align*}
By the uniform boundedness theorem this implies
\[\sup_{T \in \mcal{F}} \norm{T} = \sup_{x \in A} \norm{\mrm{ev}_x} = \sup_{x \in A} \norm{x} < \infty \text{.}\]
\end{itemize}
\end{proof}

\begin{proposition}
Let $H$ be a Hilbert space and $T \colon H \to H$ be linear and self-adjoint. Then $T$ is continuous.
\end{proposition}

\begin{proof}
Let
\[A \ceq \set{Tx}{\norm{x} \leq 1} \text{.}\]
We know that every $f \in H^*$ is of the form $f\prs{x} = \trs{x,y}$. Hence
\begin{align*}
f\prs{A} &= \set{f\prs{a}}{a \in A}
\\&= \set{f\prs{Tx}}{\norm{x} \leq 1}
\\&= \set{\trs{Tx, y}}{\norm{x} \leq 1}
\\&= \set{\trs{x, Ty}}{\norm{x} \leq 1} \text{.}
\end{align*}
By Cauchy-Schwarz,
\[\abs{\trs{x,Ty}} \leq \norm{x} \norm{Ty} \leq \norm{Ty}\]
so $f\prs{A}$ is bounded.
\\
By the uniform boundedness theorem it follows that $A$ is bounded, hence $T$ is continuous.
\end{proof}

\subsection{Applications to Harmonic Analysis}

\begin{definition}
Given $f \in \mcal{C}\prs{\mbb{T}}$ and define
\[\hat{f}\prs{n} = \frac{1}{2 \pi} \int_0^{2 \pi} f\prs{x} e^{-i n x} \diff x \text{.}\]
The Fourier series of $f$ is
\[\sum_{n \in \mbb{Z}} \hat{f}\prs{n} e^{inx} \text{.}\]
\end{definition}

\begin{theorem}[Dirichlet]
Let $f \in \mcal{C}prs{\mbb{T}}$, it holds that
\[\sum_{n \in \mbb{Z}} \hat{f}\prs{n} e^{inx} = f\prs{x}\]
and the partial sums $S_N f = \sum_{n = - N}^N \hat{f}\prs{n} e^{inx}$ converge uniformly to $f$.
\end{theorem}

\begin{theorem}
Let
\[A \ceq \set{f \in \mcal{C}\prs{\mbb{T}}}{\forall q \in \mbb{Q} \colon \text{$\prs{S_N f\prs{q}}$ isn't bounded}} \text{.}\]
$A$ is comeagre in $\mcal{C}\prs{\mbb{T}}$.
\end{theorem}

\begin{proof}
For $q \in \mbb{Q}$ define
\[A_q \ceq \set{f \in \mcal{C}\prs{\mbb{T}}}{\prs{S_N f\prs{q}}_{N \in \mbb{N}} \text{ isn't bounded}} \text{.}\]
Then $A = \bigcap_{q \in \mbb{Q}} A_q$, so it's enough to show that each $A_q$ is comeagre. WLOG assume $q = 0$.
Define
\begin{align*}
T_n \colon \mcal{C}\prs{\mbb{T}} &\to \mbb{C} \\
f &\mapsto S_N f\prs{0} \text{.}
\end{align*}
We claim $T_n \in \mcal{C}\prs{\mbb{T}}^*$ but $\sup_{n \in \mbb{N}} \norm{T_n} = \infty$.
By uniform boundedness this would imply
\[A_0^C = \set{f}{T_n f \text{ is bounded}}\]
is meagre.

Indeed,
\begin{align*}
T_n f &=
\sum_{n = -N}^N \hat{f}\prs{n}
\\&=
\sum_{n = -N}^N \prs{\frac{1}{2 \pi} \int_0^{2 \pi} f\prs{x} e^{\pm inx} \diff x}
\\&=
\frac{1}{2 \pi} \int_0^{2 \pi} f\prs{x} \cdot \underset{D_N\prs{x}}{\underbrace{\prs{\sum_{n = -N}^N e^{inx}}}} \diff x
\\&=
\frac{1}{2 \pi} \int_0^{2 \pi} f\prs{x} D_N\prs{x} \diff x
\end{align*}
from which
\begin{align*}
\abs{T_n f} &\leq
\frac{1}{2 \pi} \int_0^{2 \pi} \abs{f\prs{x}} \abs{D_N\prs{x}} \diff x
\\&\leq
\norm{f} \cdot \underset{I_n}{\underbrace{\frac{1}{2\pi} \int_0^{2 \pi} \abs{D_n\prs{x}} \diff x}}
\end{align*}
so $\norm{T_n} \leq I_n < \infty$.

Actually $\norm{T_n} = I_n$ by picking $f = \sgn D_N$ and approximating it by continuous functions.
But
\begin{align*}
I_N \ceq \frac{1}{2 \pi} \int_0^{2 \pi} \frac{\sin \brs{\prs{N + \frac{1}{2}} x}}{\sin\prs{\frac{x}{2}}} \diff x \xrightarrow{n \to \infty} \infty \text{,}
\end{align*}
so $\sup_{n \in \mbb{N}} \norm{T_n} = \infty$ as required by the above reduction.
\end{proof}

%11.11.2020

\subsection{The Open Mapping Theorem}

\begin{definition}[Open Map]
Let $X,Y$ be topological spaces. A map $\phi \colon X \to Y$ is called \emph{open} if $\phi\prs{U}$ is open for every open $U \subseteq X$.
\end{definition}

\begin{theorem}[The Open Mapping Theorem]
Let $X,Y$ be Banach spaces and let $T \in \mscr{L}\prs{X,Y}$. If $T$ is onto, it's open.
\end{theorem}

\begin{proof}
We prove that for any open ball $B\prs{x,r}$ the set $T\prs{B\prs{x,r}}$ contains a ball around $Tx$.
Note that
\begin{align*}
T\prs{B\prs{x,r}} &= T\prs{r B\prs{0,1} + x}
&= r\cdot T\prs{B\prs{0,1}} + Tx \text{.}
\end{align*}
Hence it's enough to show that $T\prs{B\prs{0,1}}$ contains a ball around $0$.
Note that
\[Y = T\prs{X} = \bigcup_{n \in \mbb{N}_+} T\prs{B_X\prs{0,1}} \text{.}\]
By Baire's theorem
\[\exists n \in \mbb{N}_+ \colon \mrm{int} \overline{T\prs{B\prs{0,1}}} \neq \ns\]
or in other words
\[\overline{T\prs{B\prs{0,1}}} \supseteq B\prs{y,r} \text{.}\]

We have to fix three things. We want a ball around zero, we want the image of a ball of radius $1$, and we want the actual image to contain it, rather than the closure of the image.

\begin{description}
\item[Ball Around Zero \& Image of $1$-Ball:]
Write
\begin{align*}
B\prs{0,r} &= B\prs{y,r} - y
\\&\subseteq \overline{T\prs{B\prs{0,n}} - Tx}
\\&=
\overline{T\prs{B\prs{0,n} - x}}
\\&\subseteq
\overline{T\prs{B\prs{0,n+\norm{x}}}} \text{.}
\end{align*}
Dividing by $n + \norm{x}$ we get
\begin{align*}
\overline{T\prs{B\prs{0,1}}} \supseteq B\prs{0, \frac{r}{n + \norm{x}}} \eqqcolon B\prs{0,\eps} \text{.}
\end{align*}

\item[Closure:]
Note that for every $a > 0$ it holds that
\begin{equation} \label{equation:scaling_balls}
\overline{T\prs{B\prs{0,a}}} \supseteq B\prs{0,a \eps} \text{.}
\end{equation}
We show that $T\prs{B\prs{0,1}} \supseteq B\prs{0,\frac{\eps}{2}}$.
Let $y \in B\prs{0, \frac{\eps}{2}}$. By \eqref{equation:scaling_balls} with $a = \frac{1}{2}$ there's $x_1 \in B\prs{\frac{1}{2}}$ such that
\[\norm{y - Tx_1} < \frac{\eps}{4} \text{.}\]
By \eqref{equation:scaling_balls} with $\frac{1}{2^n}$ there are $x_n \in B\prs{0, \frac{1}{2^n}}$ such that
\begin{equation}\label{equation:open_mapping_approximation}
\norm{y - T\prs{\sum_{i\in[n]} x_i}} < \frac{\eps}{2^{n+1}} \text{.}
\end{equation}
Since \[\sum_{n \in \mbb{N}_+} \norm{x_n} < \sum_{n \in \mbb{N}_+} \frac{1}{2^n} = 1 < \infty\]
and $X$ is complete, we get that
$\sum_{n \in \mbb{N}_+} x_n$ converges to sum $x$.
We have
\[\norm{x} \leq \sum_{n \in \mbb{N}_+} \norm{x_n} < 1\]
so $x \in B\prs{0,1}$.

Let $n \to \infty$ in \eqref{equation:open_mapping_approximation}, we get $\norm{y - Tx} \leq 0$ so $y = Tx$.
\end{description}
\end{proof}

\begin{corollary}\label{corollary:bounded_inverse}
Let $X,Y$ be Banach spaces and $T \in \mscr{L}\prs{X,Y}$. If $T$ is a bijection, $T^{-1}$ is bounded and there's $c > 0$ such that $\norm{Tx}_Y \geq c \norm{x}_X$.
\end{corollary}

\begin{proof}
We know by the open mapping theorem that $T\prs{B\prs{0,1}} \geq T\prs{0,\eps}$ for some $\eps > 0$, or in other words $B\prs{0,1} \supseteq T^{-1}\prs{B\prs{0,\eps}}$. Then
\begin{align*}
\norm{T^{-1} x} = \norm{T^{-1}\prs{\frac{\eps}{2} \cdot \frac{x}{\norm{x}}}} \cdot \frac{2 \norm{x}}{\eps} \leq \frac{2}{\eps} \norm{x}
\end{align*}
where in the last inequality we use $\frac{\eps}{2} \frac{x}{\norm{x}} \in B\prs{0,\eps}$.
Hence
$\norm{T^{-1}} \leq \frac{2}{\eps}$. In fact one can check $\norm{T^{-1}} \leq \frac{1}{\eps}$.

For every $x \in X$ we now have
\[\norm{x} = \norm{T^{-1} T x} \leq \norm{T^{-1}} \norm{Tx} \leq \frac{2}{\eps} \norm{Tx} \text{.}\]
Hence $\norm{Tx} \geq \frac{\eps}{2} \norm{x}$.
\end{proof}

\begin{remark}
The fact that $T^{-1}$ is bounded can be shown more directly. Since $T$ is surjective, it's open so $T^{-1}$ is continuous and therefore bounded.
\end{remark}

\begin{corollary}
Let $X$ be a complete Banach space with respect to two norms $\norm{\cdot}_1$ and $\norm{\cdot}_2$.
If
\[\exists C > 0 \forall x \in X \colon \norm{x}_1 \leq C \norm{x}_2\]
then
\[\exists \tilde{C} > 0 \forall x \norm{x}_2 \leq \tilde{C} \norm{x}_1 \text{,}\]
so the norms are equivalent.
\end{corollary}

\begin{proof}
Apply \ref{corollary:bounded_inverse} to $i \colon \prs{X, \norm{\cdot}_2} \to \prs{X, \norm{\cdot}_1}$.
\end{proof}

\subsection{Application of the Open Mapping Theorem to Harmonic Analysis}

\begin{definition}[Fourier Coefficients for Functions on the Circle]
Write
\[L_1\prs{\mbb{T}} \ceq \set{f \colon \mbb{R} \to \mbb{C}}{\substack{\text{$f$ is $2\pi$-periodic}\\\int_0^{2\pi} \abs{f} \diff x < \infty}} \cong L_1\prs{\brs{0, 2 \pi}}\]
with the norm
\[\norm{f} = \frac{1}{2 \pi} \int_0^{2\pi} \abs{f\prs{x}} \diff x \text{.}\]
This is a Banach space.
For every $f \in L_1\prs{\mbb{T}}$ and $z \in \mbb{Z}$ define
\[\hat{f}\prs{n} = \frac{1}{2\pi} \int_0^{2 \pi} f\prs{x} e^{-inx} \diff x \text{.}\]
\end{definition}

\begin{fact} \label{fact:riemann_lebesgue}
\begin{enumerate}
\item If $\hat{f} \prs{n} = 0$ for every $n \in \mbb{Z}$ then $f = 0$ in $L_1$.
\item \emph{Riemann-Lebesgue}: $\hat{f}\prs{n} \xrightarrow{n\to\pm \infty} 0$.
\end{enumerate}
\end{fact}

Given $\prs{a_n}_{n \in \mbb{Z}} \subseteq \mbb{C}$ such that $a_n \xrightarrow{n \to \pm \infty} 0$ we want to ask if there's $f \in L_1\prs{\mbb{T}}$ such that $\hat{f}\prs{n} = a_n$ for every $n \in \mbb{Z}$.
It turns out that the answer is no, which we prove using the open mapping theorem.

\begin{definition}
Define
\[c_0\prs{\mbb{Z}} \ceq \set{\prs{a_n}_{n \in \mbb{Z}}}{a_n \xrightarrow{n\to\pm\infty} 0}\]
with the supremum norm.
\end{definition}

\begin{definition}
Define
\begin{align*}
\mcal{F} \colon L_1\prs{\mbb{T}} \to c_0\prs{\mbb{Z}} \\
f &\mapsto \hat{f} \text{.}
\end{align*}
\end{definition}

\begin{remark}
$\mcal{F}$ is linear. It's bounded, because
\[\abs{\hat{f}\prs{n}} \leq \frac{1}{2\pi} \int_0^{2\pi} \abs{f\prs{x} e^{-inx}} \diff x = \norm{f} \text{.}\]
It's injective by the first part of \ref{fact:riemann_lebesgue}.
If $\mcal{F}$ is onto, then by the corollary $\norm{\mcal{F}\prs{f}}_\infty \geq c\norm{f}_1$.
But, take $f = D_N = \sum_{n \in \brs{-N,N}} e^{inx}$. Then
\begin{align*}
\norm{\mcal{F}\prs{f}} = \norm{\prs{0,0,\ldots, 0, 1, 1, \ldots, 1, 1, 0, 0, 0, \ldots}} = 1 \text{.}
\end{align*}
But, $\norm{D_N}_1 \xrightarrow{N\to\infty} \infty$.
This is a contradiction, hence $\mcal{F}$ is not onto $c_0\prs{\mbb{Z}}$.
\end{remark}

\subsection{The Closed Graph Theorem}

\begin{definition}[Graph of a Map]
Let $X,Y$ be Banach spaces and let $E \leq X$. Let
\[T \colon E \to Y\]
be linear. The \emph{graph of $T$} is
\[\Gamma\prs{T} \ceq \set{\prs{x, Tx}}{x \in E} \subseteq X \times Y \text{.}\]
\end{definition}

\begin{remark}
We can define a norm on $X \times Y$ by
\[\norm{\prs{x,y}} = \norm{x} + \norm{y} \text{.}\]
$X \times Y$ with this norm is denote $X \oplus_1 Y$ or sometimes $X \oplus Y$. This is a Banach space.
\end{remark}

\begin{definition}
$T$ is called closed if $\Gamma\prs{T} \subseteq X \oplus Y$ is a closed set.
\end{definition}

\begin{proposition}
$T$ is closed iff for every $\prs{x_n}_{n \in \mbb{N}_+} \subseteq E$ such that $x_n \to x$ implies $T x_n \to y$, it holds that $x \in E$ and $y = Tx$.
\end{proposition}

\begin{proof}
\begin{itemize}
\item Assume $T$ is closed.
If $x_n \to x$ and $T x_n \to y$ then $\prs{x_n, T x_n} \to \prs{x,y}$ so $\prs{x,y} \in \Gamma\prs{T}$, so $x \in E$ and $y \in Tx$.
\item The other direction is left as an exercise.
\end{itemize}
\end{proof}

\begin{example}
Let $X=Y = \mcal{C}\brs{0,1}$ and let $E = \mcal{C}^1\brs{0,1}$.
Define
\begin{align*}
T \colon E &\to Y \\
f &\mapsto f' \text{.}
\end{align*}
Then $T$ is closed. Assume $\prs{f_n}_{n \in \mbb{N}_+} \subseteq E$ is such that $f_n \to f$ and $f_n' \to g$. Then
\[f_n\prs{x} - f_n\prs{0} = \int_0^x f_n'\prs{t} \diff t \to \int_0^x g\prs{t} \diff t\]
where the first expression converges also to $f\prs{x} - f\prs{0}$.
Hence
\[f\prs{x} = f\prs{0} + \int_0^x g\prs{t} \diff t \text{.}\]
Hence $f \in E$ and $f' = g$.

Note that $E$ is note closed (it's in fact dense) and that $T$ is not bounded.
\end{example}

\begin{theorem}
Let $X,Y$ be Banach spaces. A closed map $T \colon X \to Y$ is continuous.
\end{theorem}

\begin{proof}
$T$ is closed, hence $\Gamma\prs{T}$ is closed in $X \oplus Y$ and is therefore a Banach space. Let $\pi_X, \pi_Y$ be the projections from $\Gamma$ to $X$ and to $Y$. $\pi_X$ is a continuous bijection so $\pi_X^{-1}$ is continuous by \ref{corollary:bounded_inverse}. Then
$T = \pi_Y \circ \pi_X^{-1}$ is continuous as a composition of continuous maps.
\end{proof}

\subsection{Projections and Quotient Spaces}

\begin{definition}[Projection]
A \emph{projection} is a linear map $P \colon X \to X$ such that $P^2 = P$.
\end{definition}

\begin{proposition}
Given a projection $P \colon X \to X$ we have $X = \im P \oplus \ker P$.
\end{proposition}

\begin{proof}
Let $x \in X$, we can write $x = \prs{x - Px} + Px$ where $Px \in \im\prs{P}$ and $x - P_x \in \ker \prs{P}$. If $x \in \im\prs{P} \cap \ker\prs{P}$ there's $y \in X$ such that $x = Py$ then
\[0 = Px = P^2 y = Py = x\]
so $x = 0$, so the sum is direct.
\end{proof}

\begin{remark}
If $x = e + f$ where $e \in \im P$ and $f \in \ker P$ we get
\[Px = Pe + Pf = Pe = e \text{.}\]
\end{remark}

\begin{definition}[Projection onto a Subspace]
Let $P \colon X \to X$ be a projection, let $E = \im P$ and $F = \ker P$. We say $P$ is \emph{the projection onto $E$ parallel to $F$}.
\end{definition}

\begin{definition}[Complemented Subspace]
A closed subspace $E \leq X$ of a Banach space is called \emph{complemented} if there exists $F \leq X$ closed such that $X = E \oplus F$.
\end{definition}

\begin{theorem}
Let $X$ be a Banach space.
For a closed $E \leq X$ the following are equivalent.
\begin{enumerate}
\item $E$ is complemented.
\item There is a continuous projection $P \colon X \to E$.
\end{enumerate}
\end{theorem}

\begin{proof}
\begin{description}
\item[2 $\implies$ 1:]
Take $F = \ker P$ which is closed, and $X = E \oplus F$ since $E = \im P$ and $F = \ker P$.
\item[1 $\implies$ 2:]
Assume $X = E \oplus F$ where $E,F$ are closed subspaces. Take $P$ to be the projection onto $E$ parallel to $F$.

We show that $P$ is closed. Assume $\prs{x_n}_{n \in \mbb{N}_+}$ is such that $x_n \to x$ and $P x_n \to y$. Since $E$ is closed, $y \in E$. But,
\[x-y = \lim_{n\to\infty} \prs{x_n - P x_n} \in F\]
since $F$ is closed.
Hence
\[x = y + \prs{x-y}\]
where $y \in E$ and $\prs{x-y} \in F$.
So, by definition, $Px = y$.
Hence $P$ is closed.

By the closed graph theorem, $P$ is then continuous.
\end{description}
\end{proof}

\begin{fact}
\begin{enumerate}
\item $c_0$ is not complemented in $\ell_\infty$.
\item \emph{Lindenstaruss-Tzafriri}: Every closed subspace of $X$ is complemented if and only if $X$ is isomorphic (in the sense that there's a bijection $T \colon X \to H$ such that $c\norm{x} \leq \norm{Tx} \leq C\norm{x}$ for constants $c,C > 0$) to a Hilbert space.
\end{enumerate}
\end{fact}

%TODO fill in third hour

\end{document}